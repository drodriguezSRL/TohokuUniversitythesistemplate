\chapter{Introduction}
\label{chap:intro}
\epigraph{\footnotesize{Part of the content of this chapter was first published on a paper titled \textit{``Why has no one done this before? A review of firsts''}, Journal of Esoteric Questions, 1969, 7(20), 20-17.}}

\lipsum[1]

\section{First section of the introduction}
\label{sec:addref}

\lipsum[2]

\begin{figure}[!h]
\centering
\includegraphics[width=0.6\textwidth]{image.jpg}
\caption{Add a citation here (Source: NASA\slash JPL\slash DLR).}
\label{fig:image}
\end{figure}

\subsection{Wow here goes a subsection}
\label{subsec:addref}

\lipsum[4]

\subsection{Citations}
\label{subsec:citations}

Here is how you can refer to a chapter, such as Chapter~\ref{chap:intro}, or a figure, such as Fig.~\ref{fig:image}. If you would like to cite a reference on your bibliography there are two options. First option is an in-text citation such as this one here by \citet{Kubota2005}. The second one is a parenthetical citation such as this one \citep{Giordano2009}.

\subsection{Subfigures}
\label{subsec:addref}

In the case of subfigures, use the following code for the outcome displayed in Fig.~\ref{fig:subfigures}. 

\begin{figure}[!h]
\centering
\subfigure[Image one.]
{\includegraphics[width=0.4\linewidth]{image.jpg}}
\hspace{10mm}%\hfill 
\subfigure[Image two.]
{\includegraphics[width=0.4\linewidth]{image.jpg}}
\caption{Example of two horizontal subfigures.}
\label{fig:subfigures}
\end{figure}

\subsection{Lists}
\label{subsec:lists}

This is how to add a list. You can change the numbering style using the following code. 

\renewcommand{\theenumi}{\roman{enumi}}
\begin{enumerate}
\item Here goes the first item of the list;
\item And the second item of the list; 
\item And the third item of the list;
\item You get the idea. 
\end{enumerate}

\subsection{Equations}
\label{subsec:eqs}

Here is how you input an equation in LaTeX. 

\begin{equation}\label{eqn:addref}
p(z) = \left( \frac{k_c}{b} + k_{\phi} \right) z^n.
\end{equation}

If you would like to input in-text math, there is a very simple way. For instance, you can say that $E=mc^2$ is a very popular equation but how many people you think could explain it.


\subsection{Tables}
\label{subsec:tables}

Tables can be formatted using the following code. 

\begin{table}[!hbt]
\begin{tabular}{p{1.5cm}p{1.8cm}p{2cm}p{1.7cm}p{1.7cm}p{2cm}}
\hline
Case Study & Wheel Type & Diameter [mm] & Width [mm] & No. Grousers & Soil Type \\
\hline
1 & \textbf{Rigid} & 250 & 112 & 12 & RMCS14 \\
2 & \textbf{Flexible} & 250 & 112 & 12 & RMCS14 \\
3 & Rigid & 250 & \textbf{50} & 12 & RMCS14 \\
4 & Rigid & 250 & 112 & 12 & \textbf{RMCS13} \\ 
\hline
\end{tabular}
\caption{Example of how to format a simple table.}
\label{table:cases}
\end{table}